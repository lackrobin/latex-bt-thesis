\documentclass[12pt, a4paper, twoside]{report}
\usepackage[utf8]{inputenc}
\usepackage{graphicx}
\usepackage{longtable}
\usepackage{pgf-pie} 
\usepackage{xr}
\graphicspath{{images/}}
\usepackage[a4paper,width=150mm, top=25mm, bottom=25mm, bindingoffset=6mm]{geometry}
\usepackage{fancyhdr}
\usepackage[automake]{glossaries}
\makeglossaries
\newglossaryentry{User}
{
    name=User,
    description={Eine person welche die Webseite besucht die analysiert wird.},
    plural={Users}
}


\pagestyle{fancy}
\fancyhead{}
\fancyhead[RO,LE]{Analyse von Webanalyse-Tools für den Einsatz im Verwaltungskontext}
\setlength{\headheight}{15pt}
\fancyfoot{}
\fancyfoot[LE,RO]{\thepage}
\fancyfoot[LO,RE]{\input{tex/text-resources/author}}
\renewcommand{\headrulewidth}{0.2pt}
\renewcommand{\footrulewidth}{0.2pt}
\usepackage{pgfplots}
\pgfplotsset{width=6.6cm,compat=1.7}

\usepackage[style=authoryear, backend=biber]{biblatex}
\addbibresource{references.bib} 
\listfiles

\usepackage{parskip}

\usepackage{url} 
\expandafter\def\expandafter\UrlBreaks\expandafter{\UrlBreaks%  save the current one
  \do\a\do\b\do\c\do\d\do\e\do\f\do\g\do\h\do\i\do\j%
  \do\k\do\l\do\m\do\n\do\o\do\p\do\q\do\r\do\s\do\t%
  \do\u\do\v\do\w\do\x\do\y\do\z\do\A\do\B\do\C\do\D%
  \do\E\do\F\do\G\do\H\do\I\do\J\do\K\do\L\do\M\do\N%
  \do\O\do\P\do\Q\do\R\do\S\do\T\do\U\do\V\do\W\do\X%
  \do\Y\do\Z}
\setlength{\parskip}{1em}

\makeatletter
\def\@makechapterhead#1{%
  \vspace*{50\p@}%
  {\parindent \z@ \raggedright \normalfont
    \interlinepenalty\@M
    \Huge\bfseries  \thechapter.\quad #1\par\nobreak
    \vskip 40\p@
  }}
\makeatother

\title{Analyse von Webanalyse-Tools für den Einsatz im Verwaltungskontext}
\author{\input{tex/text-resources/author}}
\date{\today}


\begin{document}
\sloppy
\begin{titlepage}
    \textbf{Berner Fachhochschule}\\
    Departement Wirtschaft 
    \par\noindent\rule{\textwidth}{0.4pt}
    \begin{center}
        \Huge
        \textbf{Analyse von Webanalyse-Tools für den Einsatz im Verwaltungskontext}

        \vspace{0.5cm}

        \LARGE
        Datenschutzkonformer Einsatz von Webanalysetools im Verwaltungskontext

        \vspace{1.5cm}
        Bachelor Thesis\\
        \textbf{Vorgelegt von: \input{tex/text-resources/author}}

        \vfill
        

        \vspace{2cm}

        \includegraphics[width=0.8\textwidth]{BFH.png}
        \large
        \begin{tabbing}
            \qquad \= Eingereicht im Studiengang: \qquad \qquad \qquad  \= \input{tex/text-resources/department}\\
            \> Ort der Einreichung:  \> \input{tex/text-resources/citycountry}\\
            \> Erstgutachterin: \> Katinka Weissenfeld \\
            \> Zweitgutachter: \> Sebastian Höhn \\
            \> Datum: \> 22.05.2020
        \end{tabbing}
        
    \end{center}
\end{titlepage}

\thispagestyle{plain}
\begin{center}
    \Large
    \textbf{Analyse von Webanalyse-Tools für den Einsatz im Verwaltungskontext}
    
    \vspace{0.4cm}
    \large
    Datenschutzkonformer Einsatz von Webanalysetools im Verwaltungskontext

    \vspace{0.4cm}
    \textbf{\input{tex/text-resources/author}}

    \vspace{0.4cm}
    \textbf{Management Summary}

\end{center}
Webanalysetools sind aus dem Internet kaum mehr wegzudenken. Auch bei den Webseiten der öffentlichen Verwaltungen kommen Webanalysetools vermehrt zum Einsatz. Da Webanalysetools durch das Sammeln, Analysieren und Auswerten von Benutzerdaten Einsichten in die Benutzung der Webseite geben, ist besonders darauf acht zu geben keine Datenschutzgesetze zu verletzten. Dies gilt speziell im Kontext der öffentlichen Verwaltungen. Die öffentlichen Verwaltungen unterliegen den kantonalen Datenschutzgesetzen und können Ŵebanalyse nur unter deren restriktiver Gesetzgebung ausüben.

Das Ziel der Thesis ist es aufzuzeigen, wie sich die Anforderungen der öffentlichen Verwaltungen, insbesondere im Bezug auf den Datenschutz, von den Anforderungen anderer Organisationen abheben. Auch wird gezeigt, welche Webanalysetools sich am Markt befinden und welche Funktionen diese anbieten. Auf Basis dessen wird eine Methode aufgezeigt, wie unter den Anforderungen der öffentlichen Verwaltungen ein Tool ausgewählt werden kann.

Die Anforderungen an Webanalysetools bei den öffentlichen Verwaltungen sind besonders im Bereich des Datenschutzes strikt. Eine zusätzliche Komplexität durch die unterschiedlichen Gesetze der Kantone erschwert die Konformität mit dem Datenschutzgesetz. Die Arbeit zeigt im Kontext der Webanalyse die Gesetze der einzelnen Kantone auf und erläutert wo sich die Grenzen zur Datenschutzkonformität befinden. Dabei wird ersichtlich, dass sich die Rechtslage bezüglich Webanalyse von Kanton zu Kanton unterscheiden kann.

Die Analyse der Tools am Markt basiert sowohl auf den Ergebnissen einer Umfrage bei den öffentlichen Verwaltungen als auch den Ergebnissen von Datanyze, einem Online-Dienst für Marktforschung. Aufgrund dessen wird eine Auswahl an Tools getroffen, welche genauer analysiert werden. Mittels der Ergebnisse der Umfrage und der Recherche der Tools auf deren offiziellen Webseiten werden die Funktionen der Tools ermittelt. Anhand der wissenschaftlichen Entscheidungsmethode AHP wird anschliessend dargelegt, wie ermittelt werden kann, welches Tool sich unter den Anforderungen der öffentlichen Verwaltungen am besten eignet. Da die Berechnung zur Eignung eines Tools auf den Anforderungen der jeweiligen Verwaltung basiert, beinhaltet die Arbeit ein Spreadsheet, welches die Berechnung automatisiert. Dadurch können Verwaltungen nun zugeschnitten auf ihre Anforderungen eine Berechnung durchführen und das sich am besten geeignete Tool zur Webanalyse ermitteln.

 
\renewcommand{\contentsname}{Inhaltsverzeichnis}
\renewcommand{\listfigurename}{Abbildungsverzeichnis}

\renewcommand{\listtablename}{Tabellenverzeichnis}

\renewcommand{\figurename}{Abbildung}
\renewcommand{\tablename}{Tabelle}
\tableofcontents

\glsaddall


\printglossary[title=Glossar, toctitle=Glossar,nonumberlist]

\vspace{2cm}
\textbf{\underline{Abkürzungen}}
\begin{description}
    \item[AHP] Analytical Hierarchical Process
    \item[ASP] Application Service Provider
    \item[CMS] Content Management System
    \item[DSGVO] Datenschutz-Grundverordnung 
    \item[GDPR] General Data Protection Regulation
    \item[IP] IP-Adresse / Internet-Protocol-Adresse
    \item[SaaS] Software as a Service
  \end{description}

\chapter{Einleitung}

\section{Ausgangslage}
Es gibt auf dem Markt unzählige Webanalysetools, sodass sich die Frage nach dem passenden Tool oftmals stellt. Auch im Verwaltungskontext stellt man sich häufig diese Frage. Hinzu kommen die Bedenken des Datenschutzgesetzes, welches durch das Einsetzten gewisser Tools tangiert werden könnte. Des Weiteren sind sich Verwaltungen unklar darüber, was beim Einsatz solcher Tools analysiert werden kann und wo dabei die Grenzen der Interpretation liegen.

\section{Problemstellung}

Durch diese Gegebenheiten sind heute bei den Verwaltungen Webanalysetools vereinzelt im Einsatz. Bei der Anschaffung eines Webanalysetools im Verwaltungskontext besteht aufgrund der zusätzlichen Kriterien wie Datenschutz und Datensicherheit eine gewisse Unsicherheit welche Tools sich für einen Einsatz eignen würden und wo sie einen Nutzen erbringen können. Auch ist es unklar, welche Tools am Markt verfügbar sind und welche dieser Tools die Kriterien einer Verwaltung am besten abdecken können. Fehlend ist eine gute Methode für das Ermitteln des sich am besten eignenden Tools.

\section{Methodik}
\subsection{Zweck und Ziel}
Diese Bachelorarbeit soll einen Überblick verschaffen, welche Webanalysetools im Verwaltungskontext eingesetzt werden, was deren Vor- und Nachteile sind und ob die Tools den verwaltungstechnischen Anforderungen entsprechen. Es soll ein gangbarer Weg aufgezeigt werden, wie Verwaltungen das richtige Webanalysetool finden, um bei sich einsetzen zu können, ohne dabei gegen das Datenschutzgesetz zu verstossen. Die Analyse soll auch ein Grundverständnis dafür schaffen, welche Tools am Markt verfügbar sind, was Hindernisse für deren Einsatz im Verwaltungskontext sein können und und inwiefern die Webanalyse von den Datenschutzgesetzen betroffen ist.


\subsection{Vorgehen}

Als erster Schritt befasst sich der Autor mit der Recherche der Thematik und Fachwissen. Hierbei liegt der Fokus auf den Domänen der Webanalysetools sowie dem Verwaltungskontext. Es muss ein sowohl Grundverständnis für die Funktionsweise von Webanalyse als auch ein Basiswissen der gängigsten Tools aufgebaut werden. Ebenfalls muss Klarheit im Bezug auf die Rechtslage des Verwaltungskontextes beim Einsetzen von Webanalysetools geschaffen werden.

Zweitens wird mittels einer Erhebung bei Verwaltungen Folgendes analysiert werden:

\begin{enumerate}
    \item Welche Webanalysetools befinden sich bereits bei Verwaltungen im Einsatz?
    \item Was sind die Vor- und Nachteile dieser Tools?
    \item Welche Funktionen werden verwendet und welche wünscht man sich?
\end{enumerate}

Schlussendlich wird Aufgrund dessen ein Bewertungsraster sowie ein Vorgehen für das Anschaffen und Einsetzen eines Analysetools erarbeitet. 

\subsection{Validität und Zuverlässigkeit}
Die Validität der Daten, speziell Informationen im Bezug auf Features der einzelnen Tools, gelten für den Zeitraum des Verfassens der Arbeit. Es kann durchaus sein, dass sich im Verlaufe der Zeit die Eigenschaften gewisser Webanalysetools ändern werden. 

Des Weiteren wurde bei der Recherche darauf acht gegeben, dass im Falle Webseiten als Quellen verwendet werden sollten, Quellen aus erster Hand zu verwenden. Für die Spezifikationen der Analysetools wir auf die offizielle Herstellerseite referenziert.

Aufbereitete Daten, für die Analyse des Marktes, wurde aus der Datenbank von Datanyze \parencite{datanyzeSwitzerlandWebanalytics} herbeigezogen. Gemäss Datanyze \parencite{datanyzeFAQ} werden die Daten durch Analysieren von Datenpunkten täglich aktualisiert und ausschliesslich In-House betrieben.

Die Umfrage fiel mit 23 Teilnehmer eher schwach aus. Die Teilnehmer der Umfrage sind jedoch grösstenteils die Webmaster der jeweiligen Webseite der Verwaltung, was für qualitative Antworten spricht.

Die Berechnungen zur Selektion eines Tools dienen lediglich als Beispiel. Die Datengrundlage ist nicht aussagekräftig genug. Deshalb sollte die Berechnung mit eigenen Zahlen durchgeführt werden. 

Für die Recherche der Literatur wurden ausschliesslich über Google-Scholar auffindbare Referenzen herbeigezogen, um die Qualität der Quellen nachweisen zu können.




\section{Aufbau der Arbeit}



\chapter{Hauptteil}
\subsection{Webanalysetools und das Kantonale Datenschutzgesetz}
Da sich Webanalysetools 
\subsection{}

\subsection{Marktangebot - Webanalysetools}

Webanalysetools lassen sich aufgrund von verschiedenen Charakteristiken Gruppieren.
Eine übliche Gruppierung basiert auf der Methode der Datensammlung. Diese lässt sich im wesentlichen auf vier verschiedene Arten durchführen \parencite{nakatani2011toolselectionmethod}:

\begin{enumerate}
    \item Web Beacon - Bei Web Beacons handelt es sich um kleine Bilddateien, welche sich auf einer Webseite eingefügt werden können und per Request angefordert werden. Sie ermöglichen es, durch das zählen der Downloads dieses Beacons, die Anzahl Hits von Seiten oder sogar Unterseiten aufzuzeichnen. Web Beacons werden in der Praxis oft als Teil von Page Tagging eingesetzt. \parencite[S. 173]{nakatani2011toolselectionmethod}.
    \item Packet Sniffing - Durch das Abfangen und analysieren der Webpakete auf dem Webserver, können mit Packet Sniffing Daten gesammelt werden. Packet Sniffing findet in der Praxis hauptsächlich Anwendung bei multivariatem Testen \parencite[S. 4]{waisberg2009webShort}.
    \item Transaktionsloganalyse - Bei dieser Methode werden die Log Dateien auf dem Webserver analysiert. Transaktionsloganalyse ist eine in der Praxis nach wie vor viel verwendete Methode, um Daten über Seitenaufrufe zu sammeln.\parencite[S. 173]{nakatani2011toolselectionmethod}. Die Tiefe der Daten die jedoch gesammelt werden kann hält sich jedoch in Grenzen. 
    \item Page Tagging - \parencite[S. 173]{nakatani2011toolselectionmethod}
  \end{enumerate}

\subsection{}
\subsection{}
\subsection{}
\subsection{}


Results - \ref{fig: a black image}

\begin{figure}[h]
    \centering
    \includegraphics{BFH.png}
    \caption{Example image}
    \label{fig: a black image}
\end{figure}



\printbibliography[title={Literaturverzeichnis}, heading=bibintoc]

\listoffigures

\listoftables

\appendix
\chapter{Umfrage}
\label{appendix:umfrage}
\url{https://drive.google.com/open?id=1Zh3EiIUqs0ooGNPBS95_vReeZYnbmZIf}
\chapter{E-Mail i-web}
\label{appendix:emailiweb}

Guten Tag Herr Lack

Entschuldigen Sie die Wartezeit, ich musste dies noch von unserer Kommunikationsabteilung prüfen lassen.

Gerne beantworte ich Ihre Fragen wie folgt:

Welche Funktionen bietet das Analyse- und Auswertungstool (Welche Metriken werden gesammelt, was lässt sich auswerten, lassen sich die Rohdaten exportieren, etc. )
Die integrierte Business-Statistik zeichnet die Zugriffe eindeutiger Benutzer/-innen pro Modul auf (meistbesuchte Module und meistbesuchte Objekte pro Modul).. Auch detaillierte Informationen zum Benutzerverhalten (Herkunft, Suchbegriffe, Nutzung eGov-Applikationen usw.) werden aufgezeichnet. Dazu kommen Informationen zur Infrastruktur (Geräte und Betriebssysteme), Loyalität (Aufenthaltsdauer, Anzahl besuchte Seiten) usw.

Wie wird der Datenschutz gewährleistet?
Die Datenschutzkonformität der Datensammlung wird sichergestellt durch Anonymisierung und “Einwilligungs-Knopf

Falls Sie aussagen zum Preis machen können würde mich dies ebenfalls interessieren.
Der Preis wird im Grundsystem abgedeckt.
Ich hoffe Ihnen hiermit geholfen zu haben und wünsche Ihnen eine erfolgreiche Arbeit.

Freundliche Grüsse

\chapter{Berechnungen mit AHP}
\label{appendix:berechnungAHP}

\url{https://docs.google.com/spreadsheets/d/135jtPmEe9L6FWIdjgN2ZseCkUZ6aVKlTdd6l_VNYOQE/edit?usp=sharing}

\end{document}


\thispagestyle{plain}
\begin{center}
    \Large
    \textbf{Analyse von Webanalyse-Tools für den Einsatz im Verwaltungskontext}
    
    \vspace{0.4cm}
    \large
    Datenschutzkonformer Einsatz von Webanalysetools im Verwaltungskontext

    \vspace{0.4cm}
    \textbf{\input{tex/text-resources/author}}

    \vspace{0.4cm}
    \textbf{Management Summary}

\end{center}
Webanalysetools sind aus dem Internet kaum mehr wegzudenken. Auch bei den Webseiten der öffentlichen Verwaltungen kommen Webanalysetools vermehrt zum Einsatz. Da Webanalysetools durch das Sammeln, Analysieren und Auswerten von Benutzerdaten Einsichten in die Benutzung der Webseite geben, ist besonders darauf acht zu geben keine Datenschutzgesetze zu verletzten. Dies gilt speziell im Kontext der öffentlichen Verwaltungen. Die öffentlichen Verwaltungen unterliegen den kantonalen Datenschutzgesetzen und können Ŵebanalyse nur unter deren restriktiver Gesetzgebung ausüben.

Das Ziel der Thesis ist es aufzuzeigen, wie sich die Anforderungen der öffentlichen Verwaltungen, insbesondere im Bezug auf den Datenschutz, von den Anforderungen anderer Organisationen abheben. Auch wird gezeigt, welche Webanalysetools sich am Markt befinden und welche Funktionen diese anbieten. Auf Basis dessen wird eine Methode aufgezeigt, wie unter den Anforderungen der öffentlichen Verwaltungen ein Tool ausgewählt werden kann.

Die Anforderungen an Webanalysetools bei den öffentlichen Verwaltungen sind besonders im Bereich des Datenschutzes strikt. Eine zusätzliche Komplexität durch die unterschiedlichen Gesetze der Kantone erschwert die Konformität mit dem Datenschutzgesetz. Die Arbeit zeigt im Kontext der Webanalyse die Gesetze der einzelnen Kantone auf und erläutert wo sich die Grenzen zur Datenschutzkonformität befinden. Dabei wird ersichtlich, dass sich die Rechtslage bezüglich Webanalyse von Kanton zu Kanton unterscheiden kann.

Die Analyse der Tools am Markt basiert sowohl auf den Ergebnissen einer Umfrage bei den öffentlichen Verwaltungen als auch den Ergebnissen von Datanyze, einem Online-Dienst für Marktforschung. Aufgrund dessen wird eine Auswahl an Tools getroffen, welche genauer analysiert werden. Mittels der Ergebnisse der Umfrage und der Recherche der Tools auf deren offiziellen Webseiten werden die Funktionen der Tools ermittelt. Anhand der wissenschaftlichen Entscheidungsmethode AHP wird anschliessend dargelegt, wie ermittelt werden kann, welches Tool sich unter den Anforderungen der öffentlichen Verwaltungen am besten eignet. Da die Berechnung zur Eignung eines Tools auf den Anforderungen der jeweiligen Verwaltung basiert, beinhaltet die Arbeit ein Spreadsheet, welches die Berechnung automatisiert. Dadurch können Verwaltungen nun zugeschnitten auf ihre Anforderungen eine Berechnung durchführen und das sich am besten geeignete Tool zur Webanalyse ermitteln.

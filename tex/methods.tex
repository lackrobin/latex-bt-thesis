\subsection{Zweck und Ziel}
Diese Bachelorarbeit soll einen Überblick verschaffen, welche Webanalysetools im Verwaltungskontext eingesetzt werden, was deren Vor- und Nachteile sind und ob die Tools den verwaltungstechnischen Anforderungen entsprechen. Es soll ein gangbarer Weg aufgezeigt werden, wie Verwaltungen das richtige Webanalysetool finden, um bei sich einsetzen zu können, ohne dabei gegen das Datenschutzgesetz zu verstossen. Die Analyse soll auch ein Grundverständnis dafür schaffen, welche Tools am Markt verfügbar sind, was Hindernisse für deren Einsatz im Verwaltungskontext sein können und inwiefern die Webanalyse von den Datenschutzgesetzen betroffen ist.


\subsection{Vorgehen}

In einem ersten Schritt befasst sich der Autor mit der Recherche der Thematik und baut das nötige Fachwissen auf. Hierbei liegt der Fokus auf den Webanalysetools sowie dem Kontext der öffentlichen Verwaltungen. Es muss ein sowohl Grundverständnis für die Funktionsweise von Webanalyse als auch ein Basiswissen der gängigsten Tools aufgebaut werden. Ebenfalls muss Klarheit im Bezug auf die Rechtslage des Verwaltungskontextes beim Einsetzen von Webanalysetools geschaffen werden. 

In einem zweiten Schritt wird mittels einer Erhebung bei den öffentlichen Verwaltungen Folgendes analysiert werden:

\begin{enumerate}
    \item Welche Webanalysetools befinden sich bereits bei öffentlichen Verwaltungen im Einsatz?
    \item Was sind die Vor- und Nachteile dieser Tools?
    \item Welche Funktionen werden verwendet und welche wünscht man sich?
\end{enumerate}

Schlussendlich wird Aufgrund dessen ein Bewertungsraster sowie ein Vorgehen für das Anschaffen eines Analysetools erarbeitet. Dabei wird die Methode AHP verwendet. 
Die AHP Methode kann folgendermassen zusammengefasst werden \parencite[S. 176]{nakatani2011toolselectionmethod}:

\begin{enumerate}
  \item Das Entscheidungsziel klar und deutlich beschreiben. Das Entscheidungsziel dient als oberste Hierarchiestufe.
  \item Kriterien, gegen welche die Alternativen evaluiert werden, beschreiben. Die Kriterien dienen als zweite Hierarchiestufe.
  \item Unterkriterien der Kriterien festlegen. Die Unterkriterien nehmen Stufe drei, falls nötig Stufe vier der Hierarchie ein.
  \item Die unterste Stufe der Hierarchie nehmen die Alternativen ein.
  \item Die Kriterien können nun in einer Matrix paarweise gegenübergestellt werden und derer Wichtigkeit im Vergleich zueinander, im Englischen \textit{related importance}, kann mittels nummerischem Wert zwischen 1 bis 9 festgelegt werden. Dabei gilt, je höher der Wert desto höher ist die Wichtigkeit des Kriterium A im Vergleich zum Kriterium B. Durch Division des jeweiligen Wertes durch die Summe der Kolonne kann der normalisierte Eigenvektor ermittelt werden.
  \item In je einer Matrix pro Kriterium werden die Alternativen gegenübergestellt. Mit einem Wert von 1-9 wird bewertet wie gut die Alternative A gegenüber Alternative B das Kriterium erfüllt. Durch Division des jeweiligen Wertes durch die Summe der Kolonne kann der normalisierte Eigenvektor ermittelt werden.
  \item In einer Matrix, in welcher die Anzahl Zeilen der Anzahl Alternativen und die Anzahl Kolonnen der Anzahl Kriterien entspricht, kann veranschaulicht werden, welche Alternative welches Kriterium wie gut erfüllt.
  \item Anhand dieser Matrix kann die gewichtete Punktzahl ermittelt werden. Die höchste Punktzahl gilt als beste Alternative.
\end{enumerate}

\subsection{Validität und Zuverlässigkeit}
Die Validität der Daten, speziell Informationen im Bezug auf Features der einzelnen Tools, gelten für den Zeitraum des Verfassens der Arbeit. Es kann sein, dass sich im Verlaufe der Zeit die Eigenschaften gewisser Webanalysetools ändern werden. 

Des Weiteren wurde bei der Recherche darauf acht gegeben, dass Webseiten, die als Quellen verwendet werden, aus erster Hand sind. Für die Spezifikationen der Analysetools wird auf die offiziellen Herstellerseiten referenziert.

Aufbereitete Daten, für die Analyse des Marktes, wurden aus der Datenbank von Datanyze \parencite{datanyzeSwitzerlandWebanalytics} herbeigezogen. Gemäss Datanyze \parencite{datanyzeFAQ} werden die Daten durch Analysieren von Datenpunkten täglich aktualisiert und ausschliesslich In-House betrieben.

Die Umfrage fiel mit 23 Teilnehmer eher schwach aus. Die Teilnehmer der Umfrage sind jedoch grösstenteils die Webmaster der jeweiligen Webseite der Verwaltung, was für qualitativ gute Antworten spricht.

Die Berechnungen zur Selektion eines Tools mit AHP dienen lediglich als Beispiel. Die Datengrundlage ist nicht aussagekräftig genug. Deshalb sollte die Berechnung mit eigenen Zahlen durchgeführt werden. 

Für die Recherche der Literatur wurden ausschliesslich über Google-Scholar auffindbare Referenzen herbeigezogen, um die Qualität der Quellen nachweisen zu können.

\subsection{Abgrenzung}
Die Arbeit befasst sich mit Webanalysetools die im Zusammenhang mit Content-Management-Systemen funktionieren. Andere Tools werden nicht berücksichtigt. Der Kontext für die Analysen bezieht sich auf den Einsatz von Webanalysetools in öffentlichen Verwaltungen. In der Arbeit wird dies als der Verwaltungskontext beschrieben. 

Eine Kosten-Nutzen-Rechnung für den Einsatz von Webanalysetools ist nicht Teil dieser Arbeit. Es geht lediglich darum zu evaluieren, welche Webanalysetools sich für den Einsatz im Verwaltungskontext am besten eignen. Um eine Kosten-Nutzen-Rechnung zu machen, bräuchte es Konkrete Projekte, welche in Erwägung gezogen werden. Alleine anhand der Tools lassen sich die Kosten nur schwer abschätzen. Daher werden nur die Kosten des  Tools, nicht aber des Betreiben des Tools, als Kriterium in die Berechnung zur Toolauswahl aufgenommen. 
\subsection{Zweck und Ziel}

Diese Bachelorarbeit soll einen Überblick verschaffen, welche Webanalysetools im Verwaltungskontext eingesetzt werden, was deren Vor- und Nachteile sind und ob die Tools den verwaltungstechnischen Anforderungen entsprechen. Es soll ein ein gangbarer Weg aufgezeigt werden, wie Verwaltungen das richtige Webanalyse-Tool finden um bei sich einsetzen zu können, ohne dabei gegen das Datenschutzgesetz zu verstossen. Die Analyse soll auch ein Grundverständnis dafür schaffen, welche Tools am Markt verfügbar sind, was Hindernisse für deren Einsatz im Verwaltungskontext sein können und und inwiefern die Webanalyse von den Datenschutzgesetzen betroffen ist. 


\subsection{Vorgehen}

Als erster Schritt befasst sich der Autor mit der Recherche der Thematik und Fachwissen. Hierbei liegt der Fokus auf den Domänen der Webanalyse-Tools sowie dem Verwaltungskontext. Es muss ein sowohl Grundverständnis für die Funktionsweise von Webanalyse als auch ein Basiswissen der gängigsten Tools aufgebaut werden. Ebenfalls muss Klarheit im Bezug auf die Rechtslage des Verwaltungskontextes beim Einsetzen von Webanalysetools geschaffen werden.

Zweitens wird mittels einer Erhebung bei Verwaltungen folgendes analysiert werden:

\begin{enumerate}
    \item Welche Webanalyse-Tools befinden sich bereits bei Verwaltungen im Einsatz?
    \item Was sind die Vor- und Nachteile dieser Tools?
    \item Welche Funktionen werden verwendet und welche wünscht man sich?
\end{enumerate}

Schlussendlich wird Aufgrund dessen ein Bewertungsraster sowie ein Vorgehen für das Anschaffen und Einsetzen von einem Analysetools erarbeitet. 

\subsection{Validität und Zuverlässigkeit}
Die Validität der Daten, speziell Informationen im Bezug auf Features der einzelnen Tools, gelten für den Zeitraum des Verfassens der Arbeit. Es kann durchaus sein, dass sich im Verlaufe der Zeit die Eigenschaften gewisser Webanalyse-Tools ändern werden. 

Des weiteren wurde bei der Recherche darauf acht gegeben, dass im Falle Webseiten als Quellen verwendet werden sollten, Quellen aus erster Hand zu verwenden. Für die Spezifikationen der Analysetools wir auf die offizielle Herstellerseite referenziert.

Aufbereitete Daten, für die Analyse des Marktes, wurde aus der Datenbank von Datanyze \parencite{datanyzeSwitzerlandWebanalytics} herbeigezogen. Gemäss Datanyze \parencite{datanyzeFAQ} werden die Daten durch Analysieren von Datenpunkten täglich aktualisiert und ausschliesslich In-House betrieben.

Die Umfrage fiel mit 23 Teilnehmer eher schwach aus. Die Teilnehmer der Umfrage sind jedoch grösstenteils die Webmaster der jeweiligen Webseite der Verwaltung, was für qualitative Antworten spricht.

Für die Recherche der Literatur wurden ausschliesslich über Google-Scholar auffindbare Referenzen herbeigezogen, um die Qualität der Quellen nachweisen zu können.



\section{Zweck}

Diese Bachelorarbeit soll einen Gangbaren Weg aufzeigen, wie Verwaltungen das richtige Webanalyse-Tool finden um bei sich einsetzen zu können.

\section{Ziel}

Es wird ein Überblick gängiger Webanalyse-Tools mitsamt ihrer Funktionen und Eigenschaften geschaffen. Unter Berücksichtigung der verwaltungstechnischer Anforderungen wird eine Empfehlung für das Vorgehen der Tool-Auswahl abgegeben.

\section{Vorgehen}

Als erster Schritt befasst sich der Autor mit der Recherche der Thematik und Fachwissen. Hierbei liegt der Fokus auf den Domänen der Webanalyse-Tools sowie dem Verwaltungskontext.

Zweitens wird mittels einer Erhebung bei Verwaltungen folgendes analysiert werden:

\begin{enumerate}
    \item Welche Webanalyse-Tools befinden sich bereits bei Verwaltungen im Einsatz?
    \item Was sind die Kriterien die ein Webanalyse-Tool erfüllen muss?
    \item Was sind die Ziele der Webseiten die Angeboten werden und durch Webanalyse-Tools analysiert werden sollen?
    \item Was erhofft man sich durch den Einsatz von Webanalyse-Tools zu erreichen?
    \item Welche Funktionen werden verwendet und welche wünscht man sich?
    \item Unterliegen die Organisationen speziellen Datenschutzrichtlinien oder gar Datenschutzgesetzen?
\end{enumerate}

Schlussendlich wird Aufgrund dessen ein Bewertungsraster sowie ein Vorgehen für das Anschaffen und Einsetzen von einem Analysetools erarbeitet. 

\section{Validität und Zuverlässigkeit}
Die Validität der Daten, speziell Informationen im Bezug auf Features der einzelnen Tools, gelten für den Zeitraum des Verfassens der Arbeit. Es kann durchaus sein, dass sich im Verlaufe der Zeit die Eigenschaften gewisser Webanalyse-Tools ändern werden. 

Des weiteren wurde bei der Recherche darauf acht gegeben, ausschliesslich Datenquellen aus erster Hand zu verwenden. 

Für die Recherche der Literatur wurden ausschliesslich über Google-Scholar auffindbare Referenzen herbeigezogen, um die Qualität der Quellen nachweisen zu können.
 

\section{Annahmen}

\section{Abgrenzung}

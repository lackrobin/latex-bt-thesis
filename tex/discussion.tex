Die Analyse hat bestätigt, dass tatsächlich eine grosse Mengen an Webanalysetools auf dem Markt existieren. Ungefähr 129 dieser Tools werden auf dem Schweizer Markt eingesetzt, wie dies in Kapitel \ref{sec:marktangebotwebanalyse} beschrieben ist. Jedoch zeigt Kapitel \ref{sec:marktangebotwebanalyse} auch, dass der grösste Teil der Webanalysebetreiber das Tool Google Analytics einsetzen. Die Anderen Anbieter kämpfen um jedes Promille an Marktanteil, während Google Analytics über 80\% des Marktes beherrscht. Anders sieht es jedoch aus, wenn der Markt auf den Verwaltungskontext eingegrenzt wird. Die gesetzlichen Restriktionen, denen die Verwaltungen durch die kantonalen Datenschutzgesetze unterliegen, machen das Einsetzen gewisser Tools problematisch, wenn nicht gar unmöglich. Mitunter gehört zu diesen Tools Facebook Analytics. Während es nach Google Analytics den Zweitgrössten Marktanteil aufweist, wurde es in der Umfrage kein einziges mal erwähnt. Nach einer Analyse des Tools in Kapitel \ref{\subsec:FacebookAnalytics} wurde ersichtlich, dass Facebook Analytics nur geringe Massnahmen zum Schutz der Personendaten aufweist, was es für den Einsatz im Verwaltungskontext uninteressant macht.

Die Analyse soll auch ein Grundverständnis dafür schaffen, welche Tools am 
Markt verfügbar sind, was Hindernisse für deren Einsatz im Verwaltungskontext sein können

Diese Bachelorarbeit soll einen Überblick verschaffen, welche Webanalysetools im Verwaltungskontext eingesetzt werden,

was deren Vor- und Nachteile sind

und ob die Tools den verwaltungstechnischen Anforderungen entsprechen.

Es soll ein gangbarer Weg aufgezeigt werden, wie Verwaltungen das richtige Webanalysetool finden, um bei sich einsetzen zu können, ohne dabei gegen das Datenschutzgesetz zu verstossen.


\chapter{Selbständigkeitserklärung}

Ich bestätige / Wir bestätigen, die vorliegende Arbeit selbständig verfasst zu haben. Sämtliche Textstellen, die nicht von uns stammen, sind als Zitate gekennzeichnet und mit dem genauen Hinweis auf ihre Herkunft versehen. Die verwendeten Quellen (gilt auch für Abbildungen, Grafiken u.ä.) sind im Literatur- bzw. Quellenverzeichnis aufgeführt.



Unterschrift(en):

\includegraphics[width=3cm]{sig.png}

\section{Ausgangslage}
Es gibt auf dem Markt unzählige Webanalysetools, sodass sich die Frage nach dem passenden Tool oftmals stellt. Auch in den öffentlichen Verwaltungen findet man häufig diese Schwierigkeit. Hinzu kommen die Vorgaben der Datenschutzgesetze, welche durch das Einsetzten gewisser Tools tangiert werden könnten. Des Weiteren sind sich Verwaltungen unklar darüber, was beim Einsatz solcher Tools analysiert werden kann und wo dabei die Grenzen der Interpretation liegen.

\section{Problemstellung}

Heute sind bei den öffentlichen Verwaltungen Webanalysetools vereinzelt im Einsatz. Bei der Anschaffung eines Webanalysetools im Verwaltungskontext besteht aufgrund der zusätzlichen Kriterien wie Datenschutz und Datensicherheit eine gewisse Unsicherheit, welche Tools sich für einen Einsatz eignen würden und wo sie einen Mehrwert erbringen können. Auch ist es unklar, welche Tools am Markt verfügbar sind und welche dieser Tools die Kriterien einer Verwaltung am besten abdecken können. Es fehlt eine gute Methode für das Ermitteln des sich am besten eignenden Tools.

\section{Zweck und Ziel}
Diese Bachelorarbeit soll einen Überblick verschaffen, welche Webanalysetools im Verwaltungskontext eingesetzt werden, was deren Vor- und Nachteile sind und ob die Tools den verwaltungstechnischen Anforderungen entsprechen. Es soll ein gangbarer Weg aufgezeigt werden, wie Verwaltungen das richtige Webanalysetool finden, um dieses bei sich einsetzen zu können, ohne dabei gegen das Datenschutzgesetz zu verstossen. Die Analyse soll auch ein Grundverständnis dafür schaffen, welche Tools am Markt verfügbar sind, was Hindernisse für deren Einsatz im Verwaltungskontext sein können und inwiefern die Webanalyse von den Datenschutzgesetzen betroffen ist.

\newpage
\section{Methodik}
In diesem Kapitel werden die Methoden beschrieben, welche zum Verfassen der Arbeit angewendet wurden. 
\subsection{Zweck und Ziel}
Diese Bachelorarbeit soll einen Überblick verschaffen, welche Webanalysetools im Verwaltungskontext eingesetzt werden, was deren Vor- und Nachteile sind und ob die Tools den verwaltungstechnischen Anforderungen entsprechen. Es soll ein gangbarer Weg aufgezeigt werden, wie Verwaltungen das richtige Webanalysetool finden, um bei sich einsetzen zu können, ohne dabei gegen das Datenschutzgesetz zu verstossen. Die Analyse soll auch ein Grundverständnis dafür schaffen, welche Tools am Markt verfügbar sind, was Hindernisse für deren Einsatz im Verwaltungskontext sein können und und inwiefern die Webanalyse von den Datenschutzgesetzen betroffen ist.


\subsection{Vorgehen}

Als erster Schritt befasst sich der Autor mit der Recherche der Thematik und Fachwissen. Hierbei liegt der Fokus auf den Domänen der Webanalysetools sowie dem Verwaltungskontext. Es muss ein sowohl Grundverständnis für die Funktionsweise von Webanalyse als auch ein Basiswissen der gängigsten Tools aufgebaut werden. Ebenfalls muss Klarheit im Bezug auf die Rechtslage des Verwaltungskontextes beim Einsetzen von Webanalysetools geschaffen werden.

Zweitens wird mittels einer Erhebung bei Verwaltungen Folgendes analysiert werden:

\begin{enumerate}
    \item Welche Webanalysetools befinden sich bereits bei Verwaltungen im Einsatz?
    \item Was sind die Vor- und Nachteile dieser Tools?
    \item Welche Funktionen werden verwendet und welche wünscht man sich?
\end{enumerate}

Schlussendlich wird Aufgrund dessen ein Bewertungsraster sowie ein Vorgehen für das Anschaffen und Einsetzen eines Analysetools erarbeitet. 

\subsection{Validität und Zuverlässigkeit}
Die Validität der Daten, speziell Informationen im Bezug auf Features der einzelnen Tools, gelten für den Zeitraum des Verfassens der Arbeit. Es kann durchaus sein, dass sich im Verlaufe der Zeit die Eigenschaften gewisser Webanalysetools ändern werden. 

Des Weiteren wurde bei der Recherche darauf acht gegeben, dass im Falle Webseiten als Quellen verwendet werden sollten, Quellen aus erster Hand zu verwenden. Für die Spezifikationen der Analysetools wir auf die offizielle Herstellerseite referenziert.

Aufbereitete Daten, für die Analyse des Marktes, wurde aus der Datenbank von Datanyze \parencite{datanyzeSwitzerlandWebanalytics} herbeigezogen. Gemäss Datanyze \parencite{datanyzeFAQ} werden die Daten durch Analysieren von Datenpunkten täglich aktualisiert und ausschliesslich In-House betrieben.

Die Umfrage fiel mit 23 Teilnehmer eher schwach aus. Die Teilnehmer der Umfrage sind jedoch grösstenteils die Webmaster der jeweiligen Webseite der Verwaltung, was für qualitative Antworten spricht.

Die Berechnungen zur Selektion eines Tools dienen lediglich als Beispiel. Die Datengrundlage ist nicht aussagekräftig genug. Deshalb sollte die Berechnung mit eigenen Zahlen durchgeführt werden. 

Für die Recherche der Literatur wurden ausschliesslich über Google-Scholar auffindbare Referenzen herbeigezogen, um die Qualität der Quellen nachweisen zu können.




\section{Aufbau der Arbeit}

Die Arbeit ist in zwei Teile gegliedert. 
Der erste Teil, der Hauptteil, befasst sich mit den Resultaten der Recherche sowie der Umfrage und ist in mehrere Unterkapitel aufgeteilt. Die Unterkapitel geben eine kurze Einführung in die jeweilige Thematik und widmen sich anschliessend dem Aufführen und Analysieren der jeweiligen Resultate. Dabei werden ebenfalls bereits erste Schlussfolgerungen gezogen. Der erste Teil schliesst mit einer Reflexion über die Arbeit ab. 
Der zweite Teil besteht aus dem Fazit, wobei die Ergebnisse und Schlussfolgerungen, die im ersten Teil erarbeitet wurden, zusammengezogen und verknüpft werden. Es dient als Abschluss der Arbeit und beantwortet die Thesen, die in der Zielsetzung aufgestellt wurden.
\newpage
\section{Einleitung in die Thematik}
Dieses Kapitel dient als Einführung und bietet die fachlichen Grundlagen um ein Verständnis für die nachfolgende Texte entwickeln zu können. Im Detail werden die Thematik Webanalyse-Tools, sowie der Verwaltungskontext erläutert.

\subsection{Webanalyse-Tools}
Das Internet bietet beinahe unbegrenzte Möglichkeiten für das Anbieten von Inhalten. Firmen, Privatpersonen, Ämter, Politiker, Selbstständige und viele mehr Nutzen es für den Austausch, das Anbieten und auch für das Erlangen von Informationen, das Kaufen und Verkaufen von Gütern, sowie das Austauschen von Nachrichten. Bietet man im Internet einen Dienst an, ist dieser meist zugänglich für die ganze Welt. Bietet man zum Beispiel Informationen über ein Webseite an, so werden diese Informationen durch die Besucher der Webseite abgerufen. 

Nun wäre es für den Anbieter der Informationen interessant, gewisse Dinge über die Abrufe herauszufinden. Wer ruft die Informationen ab? sind es Frauen oder Männer? Alt oder Jung? Wie gelangen die User auf meine Seite und wie navigieren sie zu den Informationen die sie Benötigen?

Mit genau diesen Fragen beschäftigen sich die Webanalyse-Tools. Sie bieten durch das Messen, Sammeln und Analysieren von Daten eine vertiefte Einsicht in die Benutzung des Dienstes und zeigen dadurch Schwachstellen oder Vierbesserungspotential auf. 

Vereinfacht kann gesagt werden, dass Webanalyse-Tools den Anbieter des Dienstes dabei unterstützen seine Ziele zu erreichen \parencite[S. 56]{AnalyticsForDummies}. Diese Ziele können sich von Dienst zu Dienst unterscheiden. Bei einem Online-Shop wäre das effektive Ziel die Besucher in Käufer umzuwandeln \parencite[S. 28]{AnalyticsForDummies}. Handelt es sich bei dem Dienst um eine informative Dienstleistung, so könnte ein Ziel des Anbieters sein, den User in einen wiederkehrenden User umzuwandeln, indem man ihn beispielsweise ein Konto erstellen lässt. Abgesehen von Zielen gibt es auch Zwischenziele. Ein Beispiel hierfür wäre das abonnieren eines Newsletters, durch welchen mit dem User interagiert werden kann. 



\subsection{Verwaltungskontext}

Der Einsatz von Webanalyse-Tools im Verwaltungskontext unterscheidet sich von der Privatwirtschaft hauptsächlich durch die Ziele welche mit den jeweiligen Diensten erreicht werden sollen. Während privatwirtschaftliche Webseiten meistens einem Geschäft dienen, können Webseiten im Verwaltungskontext rein unterstützend sein.

Verwaltungen unterliegen möglicherweise Strengeren Richtlinien was das Auswählen von Tools anbelangt. Da Webanalyse-Tools meist externe Dienste sind, hat man als Anwender dieser Tools oft keinen Einfluss auf die Datenhoheit. Dies bedeutet, dass Einschränkungen durch Datenschutzgesetze einen Einfluss auf die Auswahlmöglichkeiten von Analyse-Tools hat.


\section{Ausgangslage}
Es gibt auf dem Markt unzählige Webanalyse-Tools, sodass sich die Frage nach dem Passenden Tool oftmals stellt. Auch im Verwaltungskontext stellt man sich häufig diese Frage. Hinzu kommen die Bedenken des Datenschutzgesetz, welches durch das einsetzten gewisser Tools tangiert werden könnte. Des weiteren sind sich Verwaltungen unklar darüber, was beim Einsatz solcher Tools analysiert werden kann und wo dabei die Grenzen der Interpretation liegen.

\section{Problemstellung}

Durch diese Gegebenheiten sind Heute bei den Verwaltungen Webanalyse-Tools nur vereinzelt im Einsatz. Bei der Anschaffung eines Webanalyse-Tools im Verwaltungskontext besteht zudem aufgrund der zusätzlichen Kriterien wie Datenschutz und Datensicherheit eine gewisse Unsicherheit welche Tools sich für einen Einsatz eignen Würden und wo sie einen Nutzen erbringen können. Auch ist es unklar, welche Tools am Markt verfügbar sind und welche dieser Tools die Kriterien einer Verwaltung am besten abdecken können. Fehlend ist eine gute Methode für das ermitteln des sich am besten eignenden Tools.

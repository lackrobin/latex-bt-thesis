\section{Einleitung in die Thematik}
Dieses Kapitel dient als Einführung und bietet die fachlichen Grundlagen um ein Verständnis für die nachfolgende Texte entwickeln zu können. Im Detail werden die Thematik Webanalyse-Tools, sowie der Verwaltungskontext erläutert.

\subsection{Webanalyse-Tools}
Das Internet bietet beinahe unbegrenzte Möglichkeiten für das Anbieten von Inhalten. Firmen, Privatpersonen, Ämter, Politiker, Selbstständige und viele mehr Nutzen es für den Austausch, das Anbieten und auch für das Erlangen von Informationen, das Kaufen und Verkaufen von Gütern, sowie das Austauschen von Nachrichten. Bietet man im Internet einen Dienst an, ist dieser meist zugänglich für die ganze Welt. Bietet man zum Beispiel Informationen über ein Webseite an, so werden diese Informationen durch die Besucher der Webseite abgerufen. 

Nun wäre es für den Anbieter der Informationen interessant, gewisse Dinge über die Abrufe herauszufinden. Wer ruft die Informationen ab? sind es Frauen oder Männer? Alt oder Jung? Wie gelangen die User auf meine Seite und wie navigieren sie zu den Informationen die sie Benötigen?

Mit genau diesen Fragen beschäftigen sich die Webanalyse-Tools. Sie sollen durch das analysieren von Daten eine vertiefte Einsicht in die Benutzung des Dienstes bieten und dadurch Schwachstellen oder Vierbesserungspotential aufzeigen. 

Vereinfacht kann gesagt werden, dass Webanalyse-Tools den Anbieter des Dienstes dabei unterstützen seine Ziele zu erreichen \parencite[S. 56]{AnalyticsForDummies}. Diese Ziele können sich von Dienst zu Dienst unterscheiden. Bei einem Online-Shop wäre das effektive Ziel die Besucher in Käufer umzuwandeln \parencite[S. 28]{AnalyticsForDummies}. Handelt es sich bei dem Dienst um eine informative Dienstleistung, so könnte ein Ziel des Anbieters sein, den User in einen wiederkehrenden User umzuwandeln, indem man ihn beispielsweise ein Konto erstellen lässt.

Abgesehen von diesen Zielen gibt es auch Zwischenziele, 

\subsection{Verwaltungskontext}


\section{Ausgangslage}
Es gibt auf dem Markt unzählige Webanalyse-Tools. 

\section{Problemstellung}


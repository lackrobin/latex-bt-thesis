\section{Einleitung in die Thematik}
Dieses Kapitel dient als Einführung und bietet die fachlichen Grundlagen um ein Verständnis für die nachfolgende Texte entwickeln zu können. Im Detail werden die Thematik Webanalyse-Tools, sowie der Verwaltungskontext erläutert.

\subsection{Webanalyse-Tools}
Das Internet bietet beinahe unbegrenzte Möglichkeiten für das Anbieten von Inhalten. Firmen, Privatpersonen, Ämter, Politiker, Selbstständige und viele mehr Nutzen es für den Austausch, das Anbieten und auch für das Erlangen von Informationen, das Kaufen und Verkaufen von Gütern, sowie das Austauschen von Nachrichten. Bietet man im Internet einen Dienst an, ist dieser meist zugänglich für die ganze Welt. Bietet man zum Beispiel Informationen über ein Webseite an, so werden diese Informationen durch die Besucher der Webseite abgerufen. 

Nun wäre es für den Anbieter der Informationen interessant, gewisse Dinge über die Abrufe herauszufinden. Wer ruft die Informationen ab? sind es Frauen oder Männer? Alt oder Jung? Wie gelangen die User auf meine Seite und wie navigieren sie zu den Informationen die sie Benötigen?

Mit genau diesen Fragen beschäftigen sich die Webanalyse-Tools. Sie bieten durch das Messen, Sammeln und Analysieren von Daten eine vertiefte Einsicht in die Benutzung des Dienstes und zeigen dadurch Schwachstellen oder Vierbesserungspotential auf. 

Vereinfacht kann gesagt werden, dass Webanalyse-Tools den Anbieter des Dienstes dabei unterstützen seine Ziele zu erreichen \parencite[S. 56]{AnalyticsForDummies}. Diese Ziele können sich von Dienst zu Dienst unterscheiden. Bei einem Online-Shop wäre das effektive Ziel die Besucher in Käufer umzuwandeln \parencite[S. 28]{AnalyticsForDummies}. Handelt es sich bei dem Dienst um eine informative Dienstleistung, so könnte ein Ziel des Anbieters sein, den User in einen wiederkehrenden User umzuwandeln, indem man ihn beispielsweise ein Konto erstellen lässt. Abgesehen von Zielen gibt es auch Zwischenziele. Ein Beispiel hierfür wäre das abonnieren eines Newsletters, durch welchen mit dem User interagiert werden kann. 



\subsection{Verwaltungskontext}

Der Einsatz von Webanalyse-Tools im Verwaltungskontext unterscheidet sich von der Privatwirtschaft hauptsächlich durch die Ziele welche mit den jeweiligen Diensten erreicht werden sollen. Während privatwirtschaftliche Webseiten meistens einem Geschäft dienen, können Webseiten im Verwaltungskontext rein unterstützend sein.

Verwaltungen unterliegen strengeren Richtlinien was das Auswählen von Tools anbelangt. Da Webanalyse-Tools meist externe Dienste sind, hat man als Anwender dieser Tools oft keinen Einfluss auf die Datenhoheit. Dies bedeutet, dass Einschränkungen durch kantonale Datenschutzgesetze einen Einfluss auf die Auswahlmöglichkeiten von Analyse-Tools haben.


\section{Ausgangslage}
Es gibt auf dem Markt unzählige Webanalyse-Tools, sodass sich die Frage nach dem Passenden Tool oftmals stellt. Auch im Verwaltungskontext stellt man sich häufig diese Frage. Hinzu kommen die Bedenken des Datenschutzgesetz, welches durch das einsetzten gewisser Tools tangiert werden könnte. Des weiteren sind sich Verwaltungen unklar darüber, was beim Einsatz solcher Tools analysiert werden kann und wo dabei die Grenzen der Interpretation liegen.

\section{Problemstellung}

Durch diese Gegebenheiten sind Heute bei den Verwaltungen Webanalyse-Tools vereinzelt im Einsatz. Bei der Anschaffung eines Webanalyse-Tools im Verwaltungskontext besteht aufgrund der zusätzlichen Kriterien wie Datenschutz und Datensicherheit eine gewisse Unsicherheit welche Tools sich für einen Einsatz eignen würden und wo sie einen Nutzen erbringen können. Auch ist es unklar, welche Tools am Markt verfügbar sind und welche dieser Tools die Kriterien einer Verwaltung am besten abdecken können. Fehlend ist eine gute Methode für das ermitteln des sich am besten eignenden Tools.

\section{Methodik}
\subsection{Zweck und Ziel}
Diese Bachelorarbeit soll einen Überblick verschaffen, welche Webanalysetools im Verwaltungskontext eingesetzt werden, was deren Vor- und Nachteile sind und ob die Tools den verwaltungstechnischen Anforderungen entsprechen. Es soll ein gangbarer Weg aufgezeigt werden, wie Verwaltungen das richtige Webanalysetool finden, um bei sich einsetzen zu können, ohne dabei gegen das Datenschutzgesetz zu verstossen. Die Analyse soll auch ein Grundverständnis dafür schaffen, welche Tools am Markt verfügbar sind, was Hindernisse für deren Einsatz im Verwaltungskontext sein können und inwiefern die Webanalyse von den Datenschutzgesetzen betroffen ist.


\subsection{Vorgehen}

Als erster Schritt befasst sich der Autor mit der Recherche der Thematik und Fachwissen. Hierbei liegt der Fokus auf den Domänen der Webanalysetools sowie dem Verwaltungskontext. Es muss ein sowohl Grundverständnis für die Funktionsweise von Webanalyse als auch ein Basiswissen der gängigsten Tools aufgebaut werden. Ebenfalls muss Klarheit im Bezug auf die Rechtslage des Verwaltungskontextes beim Einsetzen von Webanalysetools geschaffen werden. 

Zweitens wird mittels einer Erhebung bei Verwaltungen Folgendes analysiert werden:

\begin{enumerate}
    \item Welche Webanalysetools befinden sich bereits bei Verwaltungen im Einsatz?
    \item Was sind die Vor- und Nachteile dieser Tools?
    \item Welche Funktionen werden verwendet und welche wünscht man sich?
\end{enumerate}

Schlussendlich wird Aufgrund dessen ein Bewertungsraster sowie ein Vorgehen für das Anschaffen eines Analysetools erarbeitet. Dabei wird die Methode AHP verwendet. 
Die AHP Methode kann folgendermassen zusammengefasst werden \parencite[S. 176]{nakatani2011toolselectionmethod}:

\begin{enumerate}
  \item Das Entscheidungsziel klar und deutlich beschreiben. Das Entscheidungsziel dient als oberste Hierarchiestufe.
  \item Kriterien gegen welche die Alternativen evaluiert werden beschreiben. Die Kriterien dienen als zweite Hierarchiestufe.
  \item Unterkriterien der Kriterien festlegen. Die Unterkriterien nehmen Stufe drei, falls nötig Stufe vier der Hierarchie ein.
  \item Die unterste Stufe der Hierarchie nehmen die Alternativen ein.
  \item Die Kriterien können nun in einer Matrix paarweise gegenübergestellt werden und derer Wichtigkeit im Vergleich zueinander, im Englischen \textit{related importance}, kann mittels Nummerischem wert zwischen 1 bis 9 festgelegt werden. Dabei gilt, je höher der Wert, desto höher ist die Wichtigkeit des Kriterium A im Vergleich zum Kriterium B. Durch Division der des jeweiligen Wertes durch die Summe der Kolonne kann der normalisierte Eigenvektor ermittelt werden.
  \item In je einer Matrix pro Kriterium werden die Alternativen gegenübergestellt. Mit einem Wert von 1-9 wird bewertet wie gut die Alternative A gegenüber Alternative B das Kriterium erfüllt. Durch Division der des jeweiligen Wertes durch die Summe der Kolonne kann der normalisierte Eigenvektor ermittelt werden.
  \item In einer Matrix, in welcher die Anzahl Zeilen der Anzahl Alternativen und die Anzahl Kolonnen der Anzahl Kriterien entspricht, kann veranschaulicht werden, welche Alternative welches Kriterium wie gut erfüllt.
  \item Anhand dieser Matrix kann die gewichtete Punktzahl ermittelt werden. Die höchste Punktzahl gilt als beste Alternative.
\end{enumerate}

\subsection{Validität und Zuverlässigkeit}
Die Validität der Daten, speziell Informationen im Bezug auf Features der einzelnen Tools, gelten für den Zeitraum des Verfassens der Arbeit. Es kann sein, dass sich im Verlaufe der Zeit die Eigenschaften gewisser Webanalysetools ändern werden. 

Des Weiteren wurde bei der Recherche darauf acht gegeben, dass im Falle Webseiten als Quellen verwendet werden sollten, Quellen aus erster Hand zu verwenden. Für die Spezifikationen der Analysetools wir auf die offizielle Herstellerseite referenziert.

Aufbereitete Daten, für die Analyse des Marktes, wurde aus der Datenbank von Datanyze \parencite{datanyzeSwitzerlandWebanalytics} herbeigezogen. Gemäss Datanyze \parencite{datanyzeFAQ} werden die Daten durch Analysieren von Datenpunkten täglich aktualisiert und ausschliesslich In-House betrieben.

Die Umfrage fiel mit 23 Teilnehmer eher schwach aus. Die Teilnehmer der Umfrage sind jedoch grösstenteils die Webmaster der jeweiligen Webseite der Verwaltung, was für qualitative Antworten spricht.

Die Berechnungen zur Selektion eines Tools mit AHP dienen lediglich als Beispiel. Die Datengrundlage ist nicht aussagekräftig genug. Deshalb sollte die Berechnung mit eigenen Zahlen durchgeführt werden. 

Für die Recherche der Literatur wurden ausschliesslich über Google-Scholar auffindbare Referenzen herbeigezogen, um die Qualität der Quellen nachweisen zu können.




\section{Aufbau der Arbeit}

